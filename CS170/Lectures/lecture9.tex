\section{Greedy Algorithms}
\subsection{Lecture 9}

A greedy algorithm is not super mathematically well-defined property. Generally,
we can separate the problems that greedy can solve into two types of problems:
\begin{enumerate}
    \item Search problems: which try to find an object in a large set.
    \item Optimization problems: which try to find the object (or objects) in a large set that have some maximal or minimal property.
\end{enumerate}
So, we then define:
\begin{definition} [Greedy Algorithm]
    A greedy algorithm is one that builds the solution to a problem iteratively using a sequence of local (or locally optimal) choices.
\end{definition}
Note that brute-force search is not greedy; in a greedy algorithm, you generally stick to your local decisions, never looking back.

We now discuss three problems that one can solve with a greedy approach.

\subsubsection{Scheduling}
We take as input as a collection of jobs that need to be done. Each job has a time interval $[x_1, y_1], \dots, [x_n, y_n]$ where $x_i < y_i$.
Now, we want to find the maximal amount of jobs that can be done without any time conflicts (e.g. with one thread).

To be greedy, what local choice should be make? 

One idea is to greedily keep doing the shortest possible remaining job; it unfortunately does not work. Consider the jobs $[1, 10], [9, 11], [11, 20]$, where
this algorithm will choose only the middle interval as one job, but clearly the optimal solution is taking the first interval and the last one.

Instead, let us make a different choice.

\begin{algothm}[Scheduling]
    Greedily pick the next job to have the shortest $y_i$ (end time) without conflicting with already-done jobs.

    \textbf{Runtime Analysis} We can implement this by sorting by the end-times. Then take the first element $[x, y]$ and consider the next element $[x', y']$. Since $y \leq y'$
    then checking for no conflict is just checking that $x' \geq y$. Thus, we have to just do a for loop over the array to implement the above logic. The bottleneck is the sort,
    giving us $\mathcal{O}(n \log n)$.

    Now, to prove correctness for greedy algorithms, we generally use an exchange argument, where we postulate the existence of a solution strictly better
    to the output produced by the algorithm, and
    show a contradiction.

    \textbf{Proof of Correctness} Suppose our algorithm gives $J$, giving jobs $j_1, j_2, \dots, j_k$ (in sorted order by endpoint) and this not optimal. Take some optimal solution $S$ such that $|S \cap J|$
    is maximum. We will show there exists another optimal solution $S'$ such that $|S' \cap J| > |S \cap J|$, giving us a contradiction of maximality.

    Let us sort by endpoint on $S$, making $s_1, s_2, \dots, s_k, s_{k + 1}, \dots$. The $j$'s are not the same as the $s$'s because otherwise the greedy algorithm would've taken more $s$'s (since the $s$'s are not conflicting and 
    sorted by end time). Thus, there must exist some first $t$ such that $j_t \neq s_t$. Now, let us introduce a new solution $S'$ with $s_t$ is replaced by $j_t$. $j_t$'s end-time
    is earlier or at the same time as $s_t$, so there cannot be any conflicts after $j_t$ (otherwise there would've been a conflict with $s_t$) and there cannot be any conflict before $j_t$ (because the rest of the $s$'s agree with $j$,
    and there is no conflict among the $j$'s). Furthermore, $|S'| = |S|$. Thus, $S'$ is also an optimal solution; but $|S' \cap J| = |S \cap J| + 1 > |S \cap J|$, as required for contradiction.
\end{algothm}

Now, working through the example above, we will select $[1, 10]$ first, then the second interval won't be added since it's conflicting, then the third interval will be added since it's not conflicting.

\subsubsection{Huffman Encoding}
We now discuss the problem of efficient encoding. Suppose we have some alphabet $\Sigma$, $|\Sigma| = n$ (e.g. $\{A, \dots, Z\}$) and some text $w$ whose characters come
from $\Sigma$. Now there are character frequencies:
\[ freq(\sigma) = \text{Number of occurences of $\sigma$ in text} \]
Now, how can we encode the text as efficiently as possible using these characters (minimizing codeword length)?
One concern for encoding is that certain codewords may have the same word that created it. We want our encoding to be unambiguous; one way to ensure this
is using prefix-freeness.

\begin{definition} [Prefix-Freeness]
    An encoding is prefix-free if no character's encoding is a prefix of any other character's encoding.
\end{definition}

Note that any code that is prefix-free cannot be ambiguous, because the unique prefix will tell you which character to look at. Thinking in terms of bits,
the following encoding:
\[ A = 0, B = 01, C = 001, \dots \]
can be represented by a binary trie where each branch is a 0 or 1; the leaves are all characters.

Now,
we want to find the optimal prefix-free encoding. Naively, we could find all binary trees on $n$ leaves ($2n - 1$ nodes) and assignments of characters to leaves.
The number of binary trees is exponential, about $16^n$ and there $n!$ assignments. We also have to do a linear check of length to check cost; this means our total runtime is $\mathcal{O}(n!)$.
Note that even we assigned all characters to leaves greedily, this would still be an exponential-time algorithm! 

Furthermore, being fully greedy and letting the highest frequency character hang at the top of the tree is also not optimal.
In the case of equal frequency, we want a perfectly balanced tree (the best compression of 26 characters is to use 5 bits, if they are equal frequencies). 

Here is a better way:

\begin{algothm}[Huffman Encoding]
    We try to build the tree bottom-up. Note that at the lowest depth, any node $v$ has a sibling (otherwise we could delete the parent of $v$ and replace $v$ with that to have less bits).
    At these two leaves, let us place the least-frequent characters here. We claim this is optimal; suppose there was another optimal way to do things where a different pair was at the
    bottom of the tree, then you can switch the one of the least-frequent to the bottom, saving you characters in your encoding.

    To do this algorithm greedily, we simply repeat this choice over and over iteratively. To see this in action, we present an example.

    First, notice that we have to do $n - 1$ iterations. To find the smallest characters in the set,
    we can keep a min-heap, requiring 1 $\Theta(\log n)$ insertion and 2 $\Theta(\log n)$ removals. Then, building the tree is done in constant time in each iteration.
    Our total runtime is $\Theta(n \log n)$.
\end{algothm}

\begin{example}
    Consider the following frequencies on the alphabet of capital vowels:
    \[ A: 60, E: 70, I: 45, O: 50, U: 20, Y: 30 \]
    Placing the two smallest nodes $U$ and $Y$ at the bottom of the tree then creates a certain "meta-character" $\{U, Y\}$ with frequency $20 + 30 = 50$. Our frequencies then become:
    \[ A: 60, E: 70, I:45, O: 50, \{U, Y\}: 50 \]
    Now we take the next two smallest nodes ($I$ and $\{U, Y\}$) and place it at the bottom of tree, creating:
    \[ A: 60, E: 70, O: 50, \{I, U, Y\}: 95 \]
    Continuing, we have:
    \begin{align*}
        &A: 60, E: 70, O: 50, \{I, U, Y\}: 95  \\
        &E: 70, \{I, U, Y\}: 95, \{A, O\}: 110 \\
        &\{E, I, U, Y\}: 165, \{A, O\}: 110 \\
        &\{A, E, I, O, U, Y\}: 275
    \end{align*}
    Then, we've build the tree; we can traverse with DFS to then get the encoding per character.
\end{example}

\subsubsection{Set Cover}
Now we consider another problem. We are given a Universe $\{1, \dots, n\} = [n]$ and a collection $S_1, \dots, S_m$ of subsets of $[n]$.
We want to find a minimum-size subcollection to cover $[n]$. Generally, we don't expect there to be a polynomial-time algorithm to solve this.
The natural greedy approach (where we take the biggest subset every time) does not give the optimal solution (easy to see with a counterexample).

Let the optimal number of sets be $OPT$.
Feige showed that if there exists a polynomial-time algorithm for set cover, which always uses less than $OPT \times \ln n)$ sets, then
3-SAT can be solved in time $\leq n^{\log \log n}$ (which we don't believe to be true).

This makes greedy even more tantalizingly close:
\begin{theorem}
    The natural greedy approach uses $\leq OPT \times \lceil \ln n \rceil$ sets.
\end{theorem}

