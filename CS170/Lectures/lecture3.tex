\subsection{Lecture 3}

\subsubsection{Recurrences and Master Theorem}
The idea of divide-and-conquer algorithms are to
divide the input inot smaller parts, recurse on parts,
and combine the parts to build an answer.

To analyze the runtime of divide-and-conquer algorithms, it is
useful to derive the following result.

\begin{theorem} [Master Theorem]
    Suppose we have a recurrence
    \[ T(n) = a T(\frac{n}{b}) + cn^d \]
    then, we have
    \[ T(n) = \begin{cases}
        \Theta(n^d) & a < b^d \\
        \Theta(n^d \log n) & a = b^d \\
        \Theta(n^{\log_b a}) & a > b^d
    \end{cases} \]
\end{theorem}

Master's theorem can be shown by drawing a recursion tree,
and then summing up the work done in each level (proof omitted for brevity).
We can also think of the cases as symbolizing the following:

\begin{center}
\begin{tabular}{ c|c }
    Case & Interpretation \\ \hline
    $a < b^d$ & The root does most of the work \\ \hline
    $a = b^d$ & The root and the leaves do an equal amount of work \\ \hline
    $a > b^d$ & The leaves do most of the work \\ \hline
\end{tabular}
\end{center}

\subsubsection{Matrix Multiplication}
Our first example of a divide and conquer algorithm is matrix multiplication.

Consider multiplying two $n$-by-$n$ matrices $A$ and $B$.

\[ A = \begin{bmatrix}
    A_{11}& A_{12}& \ldots& A_{1n} \\
    A_{21}& A_{22} & \ldots & A_{2n} \\
    \vdots& \vdots & \ddots & \vdots \\
    A_{n1} & A_{n2} & \ldots & A_{nn}
\end{bmatrix} \]

Then the resultant $C$ has entries given by:

\[ C_{ij} = \sum_{k = 1}^n A_{ik} B_{kj} \]

The natural implementation is then to to loop this summation over $i$
and $j$. This means this will be three nested loop (a loop is needed for the summation).
In flops, this runs in $\Theta(n^3)$ operations.

We can try to break our input instead into $n / 2$-by-$n / 2$ blocks, as shown:

\[ \begin{bmatrix}
    A & B \\
    C & D
\end{bmatrix} \begin{bmatrix}
    E & F \\
    G & H
\end{bmatrix} = \begin{bmatrix}
    AE + BG & AF + BH \\
    CE + DG & CF + DH
\end{bmatrix} \]

To find the runtime of this algorithm, let us realize there are
8 multiplications (recursively) and then finally a $\Theta{n^2}$ addition at the end.
This means our recurrence is:

\[ T(n) = 8 T\qty(\frac{n}{2}) + \Theta(n^2) \]

Note that for our Master theorem setup, $8 > 2^2$, so we have
\[ T(n) = n^{\log_2{8}} = n^3 \]
so this is no better than our naive approach.

Using a similar realization to Karatsuba, Strassen in 1969 found the following:

\begin{algothm} [Strassen's Algorithm]
    Consider two matrices $X$ and $Y$ which are both $n$-by-$n$.
    Break them up into block matrix form of $n/2$-by-$n/2$ matrices as follows:
    \[ X = 
    \begin{bmatrix}
    A & B \\
    C & D
    \end{bmatrix}, Y = \begin{bmatrix}
    E & F \\
    G & H 
    \end{bmatrix} \]
    Define the following:
    \begin{align*}
        P_1 &= A(F - H) \\
        P_2 &= (A + B)H \\
        &\vdots \\
        P_7 &= (A - C)(E + F)
    \end{align*}

    Then,
    \[ Z = \begin{bmatrix}
       P_5 + P_4 - P_2 + P_6 & P_1 + P_2 \\
       P_3 + P_4 & P_1 + P_5 - P_3 - P_7
    \end{bmatrix} \]

    Analyzing the runtime, we see that there are 7 $n/2$-by-$n/2$ multiplications, and some $n^2$ additions,
    meaning the total runtime is
    \[ T(n) = 7 T(n/2) + \mathcal{O}\qty(n^2) \]
    which Master theorem brings to
    \[ T(n) = \mathcal{O}\qty(n^{\log_2{7}}) \approx \mathcal{O}\qty(n^{2.81}) \]
\end{algothm}

However, Strassen has such a big constant factor that the normal $n^3$ algorithm
is still the most widely used.

\subsubsection{Sorting}
    Consider the problem of sorting a length $n$ array $A$. 

\begin{algothm}[MergeSort]

    First, we define a procedure \textsc{merge} that takes two sorted lists and merges them in linear time. To do this, we first keep a pointer on both lists that starts at the beginning of each list.
    We then compare the pointed-to elements of each list. The lesser element is then added to the output, and the pointer of the list with that element is incremented
    by one place. This keeps going until all the elements are used. We then use merge to divide-and-conquer the list as follows:
    \begin{algorithmic}
    \Function{merge-sort}{$A[1 \dots n]$} 
        \State $B \gets \Call{merge-sort}{A[1 \dots \frac{n}{2}]}$
        \State $C \gets \Call{merge-sort}{A[\frac{n}{2} + 1 \dots n]}$
        \State \Return \Call{merge}{$B, C$}
    \EndFunction
    \end{algorithmic}

    We get the following recurrence for \textsc{merge-sort}:

    \[ T(n) = 2 T\qty(\frac{n}{2}) + \Theta(n) \], which through master theorem gives us a running time of
    \[T(n) = \Theta(n \log n) \]
\end{algothm}

Another way of implementing this is a bottom up approach:

\begin{algothm} [Iterative Merge Sort]
    \begin{algorithmic}
        \Function{merge-sort-iter}{$A[1 \dots n]$}
            \State $Q \gets $ Divide $A$ into $n$ lists of size one
            \While{$Q$.size() $> 1$}
                \State $X, Y \gets Q$.pop(), $Q$.pop()
                \State $Q$.push(\textsc{merge}($X, Y$))
            \EndWhile
            \State \Return $Q$.pop()
        \EndFunction
    \end{algorithmic}
    Now we can think about the runtime of this algorithm. Think of the algorithm as running in phases. Phase 0 is when lists popped have size 1.
    Phase 1 is when lists popped have size 2. Phase $i$ is when lists popped have size $2^i$. Note that in each phase, each element is looked at exactly once. Thus,
    the total runtime must be proportional to $n \cdot \text{ number of phases}$. How many phases are there? There are $\log n$ phases, giving us
    the same $\Theta(n \log n)$ runtime!
\end{algothm}


The best way to see this is with an example:

\begin{example}
    Suppose our initial list is:
    \[ A = \begin{bmatrix}
        8 & 7 & 6 & 5 & 4 & 3 & 2 & 1
    \end{bmatrix} \]

    We then split this into sublists of size one in our $Q$ and start iterating:
    \begin{align*}
        Q &= [
            [8], [7], [6], [5], [4], [3], [2], [1]
        ] \\
        Q &= [
            [6], [5], [4], [3], [2], [1], [7, 8]
        ] \\
        &\vdots \\
        Q &= [
            [7, 8] , [5, 6] , [3, 4] , [1, 2]
        ] \\
        Q &= [[3, 4] , [1, 2] , [5, 6, 7, 8]] \\
        Q &= [[1, 2, 3, 4] , [5, 6, 7, 8]] \\
        Q &= [[1, 2, 3, 4, 5, 6, 7, 8]] 
    \end{align*}

    Our list has been sorted!
\end{example}

Can we sort faster than $n \log n$? It turns out we cannot do any better with an algorithm that uses comparisons to
sort (the problem is $\Omega(n \log n)$).

\begin{theorem} [Fastest Shorting in Comparison Model, Lower Bound]
    We will show that $\Omega(n \log n)$ comparisons are needed even if promised
    $A$ is some permutation of $\{1, \dots, n\}$ (all distinct as well).

    \begin{proof*}
        The first comparison such an algorithm might make might be:
        is $A_i < A_j$? Then we branch off into two cases for each, where we require another comparison.
        We can construct a binary tree that models the situation.


        % \begin{tikzpicture}[grow'=down]
        %     \Tree [.1
        %             [.2]
        %             [.3]]
        % \end{tikzpicture}

        First, notice that each leaf is when the algorithm terminates. Note that since every run of the algorithm with a different
        input produces a different output permutation of the input, there must be at least $n!$ leaves.
        Consider the maximum depth of this tree $T$. There are at most $2^T$ leaves, meaning that $2^T \geq n!$. This means
        \begin{align*}
            T &\geq \log(n!) \\
            T &= \Omega(n \log n)
        \end{align*}
        The last claim can be shown by realizing
        \begin{align*}
            n! &\geq \qty(\frac{n}{2})^\frac{n}{2} \\
            \log(n!) &\geq \frac{n}{2} \log\qty(\frac{n}{2})\\
            \log(n!) &\geq \frac{n}{2} \log n - \frac{n}{2} \\
            \log(n!) &= \Omega(n \log n)
        \end{align*}
    \end{proof*}
\end{theorem}

However, there are way more operations than comparisons that can be used for sorting. An example of this is counting sort:
if you have $n$ integers and all the integers have values between $1$ and $b$, then you can sort in $\Theta(n + b)$, by just
keeping a hashmap of all the elements that fit in each value "bucket" between $1$ and $b$. 

The Word RAM model: suppose your machine can store words of size $w$ and you can do any common $C$ operations. The fastest known algorithm
following this model (not just comparisons) is: $\mathcal{O}\qty(n \sqrt{\log \log n})$. This is also a randomized algorithm.

There are two types of randomized algorithm (which will be revisited). A Monte Carlo randomized algorithm is one whose
output may be incorrect with small probability. A Las Vegas algorithm is one whose runtime is fast in expectation, but may be slow
with small probability.

\subsubsection{Selection/Medians}

We now consider the problem of selection. Suppose we want to select the $k$ smallest integer in a list $A$. Without loss of generality,
assume all elements of $A$ are distinct (since we could replace $A[i]$ with the tuple $(A[i], i)$). For selection, there is Quick Select, which
we will explore later. Notably, quick select requires knowing the median in linear time. We will instead focus on that problem,
(the same as the selection problem for $k = n/2$).

\begin{algothm} [Median of Medians]
    Take an array $A$. Then break up the array into subarrays of size 5.
    Next, we will recursively compute the median of each subarray. Note that this takes constant
    time to complete since 5 is a constant. Now we have a $N / 5$ size array. We then find the median recursively of this smaller problem.
    Call this median $m_1$.

    Now, we change the array such that all the elements bigger than $m_1$ end up on the right of $m_1$,
    elements smaller than $m_1$ end up on the left, and $m_1$ is in the middle of these two parts (this only requires a linear scan).
    If $m_1$ is in position
    less than $n / 2$, then the true median sits on its right, so we can recurse on the right half.
    If $m_1$ is in position greater than $n / 2$, then the true median sits on its left, so we can recurse on the left half.
    Finally, if $m_1$ is at exactly position, $n/2$, then we have found the median.

    The claim is that at least 30\% of the elements are filtered out by comparisons to $m_1$. To show this,
    consider $m_1$ compared to the other medians. Note that it is bigger than 3 of the elements in every median it is bigger than,
    so, it is bigger than $\frac{3}{5} \cdot \frac{N}{10} = \frac{3}{10}$ of the elements. Thus, if $m_1$ is in the first half of the array,
    then it will filter out at least $\frac{3}{10}$ of the numbers. Similarly, you can make a symmetric argument that if $m_1$ is in the second half,
    then it must be less than $\frac{3}{10}$ of the numbers and thus will filter out 30\% of them. Either way, we can then produce the following recurrrence:

    \[ T(n) \leq T\qty(\frac{7n}{10}) + T\qty(\frac{n}{5}) + \Theta(n) \]

    This recurrence gives $T(n) = \mathcal{O}(n)$. We can give an inductive argument:
    \begin{proof*}
        We will show that $T(n) \leq Bn$ for sufficiently large $B$, which will imply our big-$\mathcal{O}$ runtime.

        \textbf{Base Case}: if $n$ is 1, then we just return the input, so if $B$ is greater than the time needed to return
        then the base case holds.

        \textbf{Inductive Hypothesis}: Suppose that the claim holds for $k < n$.
        
        \textbf{Inductive Step}: By the recurrence and the inductive hypothesis, we have that
        \[T(n) \leq B \frac{7n}{10} + B \frac{n}{5} + Cn\]
        where $C$ is some other constant. Now, we have
        \begin{align*}
            T(n) &\leq \qty(\qty(\frac{7}{10} + \frac{1}{5})B + C)n \\
            &\leq \qty(\frac{9}{10}B + C)n \\
            &\leq Bn
        \end{align*}
        as long as $C \leq \frac{B}{10}$. Since $C$ is fixed, we can
        set $B = 10C$ to make this true. \qed
    \end{proof*}

\end{algothm}

