\section{Graph Algorithms}

\subsection{Lecture 5}

\subsubsection{Graph Representation}

\begin{definition} [Graph]
    A graph is a pair $G = (V, E)$ where $V$ is the set of vertices and $E$
    is the set of edges. We generally call $n = |V|$ and $m = |E|$.

    If $G$ is a directed graph, then $E \subseteq V \times V$. $G$ is simple if $(a, a)$ (a self loop) is not allowed.

    If $G$ is an undirected graph, then $E$ is a set of unordered pairs from $V$. If there are no self-loops, $G$
    is simple.
\end{definition}

\begin{example} [Examples of graphs]
    Graphs are often used as convenient ways to represent data.
    \begin{enumerate}
        \item Road network where vertices are intersections, and edges are road segments connecting intersections. (Would be directed, also maybe "weighted" according to distance)
        \item Social networks where vertices are people, and edges are friendships. (Facebook would undirected, Twitter/IG would be directed)
    \end{enumerate}
\end{example}

How do we represent graphs on a computer? We will assume:
$V = \{1, \dots, n \}$. Then to store edges, we will either:
\begin{enumerate} [label=\alph*]
    \item Adjacency Matrix: $A$ is an $n$-by-$n$ matrix where 
    \[ A_{ij} = \begin{cases}
        1 & (i, j) \in E \\
        0 & (i, j) \not\in E
    \end{cases} \]
    for a weighted graph this becomes:
    \[ A_{ij} = \begin{cases}
        w_{ij} & (i, j) \in E \\
        \infty & (i, j) \not\in E
    \end{cases} \]
    \item Adjaceny List: represent $E$ as an array $B$ of linked lists.
    Then, $B[i]$ is a linked list containing all $j$ such that $(i, j) \in E$.
\end{enumerate}

We can compare the cost of these representations as follows:

\begin{tabular}{c | c | c}
    \empty & Adjancency Matrix & Adjancency List \\ \hline
    Space & $n^2$ bits & $\Theta(m + n)$ words \\ \hline
    $(u, v) \in E$? & $\mathcal{O}(1)$ & $\Theta(1 + d_u)$ \\ \hline
    Print all the neighbors of $u$ & $\Theta(n)$ & $\Theta(d_u)$
\end{tabular}

where $d_u$ is the degree of $u$, i.e. $|\{w \mid (u, w) \in E\}|$.

Note that we could also choose alternative representations. Remember that graphs are abstractions used to store data,
so there is no one-size-fits-all solution.

\subsubsection{Depth First Search (DFS)}
We discuss graph exploration, i.e. visiting all vertices in a graph.

\begin{algothm} [Depth First Search]
    \begin{algorithmic}
        \Function{dfs}{$V, E$}
            \State global clock = 1
            \State global visited = boolean[n]
            \State global preorder, postorder = int[n], int[n]
            \For{$v \in V$}
                \If {visited[v] is false}
                    explore($v$)
                \EndIf
            \EndFor
        \EndFunction

        \Function{explore}{$v$}
            \State visited[v] = true
            \State preorder[v] = clock
            \State clock = clock + 1
            \For{$(v, w) \in E$}
                \If{visited $w$ is false}
                    explore($w$)
                \EndIf
            \EndFor
            \State postorder[v] = clock
            \State clock = clock + 1
        \EndFunction
    \end{algorithmic}

    The time window between postorder[$v$] and preorder[$v$] is exactly the amount of time we spend in recursive calls from $v$.

    We have a claim about the subroutine $\textsc{explore}$. Namely, $\textsc{explore}(u)$ explores exactly:
    $\{ v \mid \exists \text{ a path from $u$ to $v$}\}$, i.e. the connected component of $u$.

    An argument for this claim is as follows:
    \begin{proof*}
        We need to show two directions.

        First, if we explored $v$, there must be a path from $u$ to $v$, since every call in the recursion is from one neighbor to another. We can construct
        the path by just following the path in the recursion tree.

        In the other direction, for the sake of contradiction, suppose there exists a reachable $v$ that doesn't get explored. Let the path from $u$ to $v$ be:
        \[ (u \to x_1 \to x_2 \to \dots \to x_r = v) \]
        Let $x_j$ be the first vertex on the path which isn't explored. Then, look at $x_{j - 1}$, which was explored, which means we looped over all of $x_{j-1}$'s neighbors.
        This means we had the opportunity to visit $x_j$, but we didn't. This means we must've visited $x_j$ and failed the if check. This is a contradiction, we visited $x_j$ and didn't.
        Thus, we must have that all reachable $v$'s are explored.
    \end{proof*}

    Now we look at the runtime of the routine. First, note that the outer for loop runs $n$ times. Then, we have to enumerate the neighbors of $u$ for all vertices $u$ in the graph.
    We only have to do this once, since we visit each vertex only once.

    This means that the total running time is:
    \[ T(n) = \Theta(n) + \sum_{u \in V} \text{time to enumerate neighbors} \]
    which depends on our graph representation. With our adjacency matrix, the second term is quadratic. In the adjacency list,
    it is just the sum of (degrees + 1) which will be exactly $2m + n$. (Since by adding degrees we double count the number of edges).

    \begin{tabular}{c | c | c}
        \empty & Adjancency Matrix & Adjancency List \\ \hline
        DFS time & $\Theta(n^2)$ & $\Theta(m + n)$
    \end{tabular}
\end{algothm}

TODO: Add a pictoral example of DFS from lecture.

There are many applications of the DFS:
\begin{itemize}
    \item Reachability - Identifying the separate connected components
    \item Articulation points - The set of vertices whose removal disconnects the graph
    \item Finding biconnected/triconnected components - if there are two/three disjoint paths between any two vertices
    \item Strongly connected components - In a directed graph, two vertices are strongly connected if there is a path from one vertex to another and from that other vertex back to the original
    \item Planarity testing - Testing if a graph is planar, i.e. you can draw it without crossings
    \item Isomorphism of planar graphs - Telling whether two planar graphs are isomorphic, i.e. we can turn one into the other with a bijection
\end{itemize}

The first set of problems we have already solved. Reachability is everything we visit in one iteration of the outer loop. The amount of connected components is the amount
of explore calls in the outer loop.

\begin{definition} [Preorder, Postorder]
    The notation $\text{pre}(u)$ and $\text{post}(u)$ denote the preorder and postorder number of a given vertex $u$ in a run of the DFS algorithm on some graph containing $u$.
\end{definition}

Now let us explore another claim about the DFS.

\begin{note}
    For some vertex $u$, the set of intervals $[\text{pre}(u), \text{post}(u)]$ are either
    pairwise nested or disjoint. The argument is that pairwise, either two are in the same connected component or different components.
    In the same component, one must have started before the other, then the recursion finished the inner one, then it must've ended. In different components,
    the intervals are disjoint.

    Let us denote previsiting a vertex $u$ as $[_u$ and postivisiting that vertex as $]_u$.
\end{note}

It helps us to classify the graph edges during a DFS.

\begin{definition} [Types of Graph Edges]
    These are the types of edges in a DFS traversal of $G$.
    \begin{enumerate}
        \item Tree edge: Traversed edge during DFS (in the DFS tree).
        \item Forward edge: Goes from an ancestor to a descendant in the DFS tree, and is not a tree edge.
        \item Back edge: Goes from a descendant to an ancestor in the DFS tree, and is not a tree edge.
        \item Cross edge: An edge that is not any of the above.
    \end{enumerate}
\end{definition}

The useful facts about these edges are as follows:

\begin{theorem}
    Suppose $(u, v) \in E$. $\text{post}(u) < \text{post}(v)$ if and only if $(u, v)$ is a back edge.

    \begin{proof*}
    The backwards part of the claim is simpler. Basically $u$ is a decendant of $v$, so the order is something like $[_v \dots [_u$, so $]_u \dots ]_v$
    must happen, i.e. the postorder numbers have the given order.

    \end{proof*}
\end{theorem}

\begin{theorem}
    $G$ has a cycle if and only if it has a back edge.

    We again take the backwards case first. By the definition of a back edge, there must be a series of tree edges from some vertex $u$ to $v$ and then a
    back edge from $v$ to $u$. This implies there is a cycle from $u$ to $u$.

    \begin{proof*}
    Now the forward part of the claim, consider the cycle:
    \[ u_0 \to u_1 to \dots \to u_k \to u_0 \]
    Take $u_i$ with the lowest postorder number. Then the edge:
    \[ (u_i, u_{(i + 1) \mod k}) \]
    goes from a higher postorder number to a lower postorder number, which means that by the previous result, this edge is a back edge, showing our claim.
    \end{proof*}
\end{theorem}
