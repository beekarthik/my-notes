\subsection{Lecture 7}

\subsubsection{Single Source Shortest Paths}

We now consider shortest paths. Suppose we are given
a directed graph $G$ and a start vertex $s$. We wish to find the shortest path from $s$ to all other vertices. More precisely,
we want two arrays:
\begin{itemize}
    \item prev$[1, \dots, n]$, where prev$[v]$ is the previous vertex to $v$ on the shortest path from $s$ to $v$.
    \item dist$[1, \dots, n]$, where dist$[v]$ is the length of the shortest path from $s$ to $v$.
\end{itemize}

Note that the prev array has enough information to give us the actual shortest path from $s$ to $v$.

There are many algorithms for single-source shortest paths. Here are a few:

\begin{itemize}
    \item Breadth-First Search - Assumes edges all have weight 1 ("unweighted")
    \item Dijkstra's Algorithm - Assumes edge weights are all non-negative.
    \item Bellman-Ford Algorithm - Arbitrary edge weights.
    \item Dynamic Programming on DAGs - Arbitrary edge weights, but $G$ must be a DAG.
\end{itemize}

Note that depth-first search does not work because going deep first may yield a longer path than some other traversal. Consider the graph:
\[ V = \{S, A, B \}, E = \{(S, A), (A, B), (S, B)\} \]
Then suppose the DFS traversal was $S, A, B$. Then the path to $B$ would seem like it's distance 2, when in reality it's 1, since the $(S,B)$ wouldn't be traversed by DFS.

\begin{algothm} [Breadth-First Search]
    Here is the main algorithm:
    \begin{algorithmic}
        \Function{BFS}{$G, s$}
            \State dist$[1\dots n] \gets \infty$
            \State prev$[1\dots n] \gets$ null
            \State vis$[1\dots n] \gets$ False
            \State Q $\gets$ queue($s$)
            \State dist$[s] \gets 0$
            \State vis$[s] \gets$ True
            \State $Q$.push($s$)
            \While{$Q$.size$>0$}
                \State $u \gets Q$.pop()
                \For{$(u, v) \in E$}
                    \If{!vis$[v]$}
                        \State vis$[v] \gets$ True
                        \State dist$[v] \gets$ dist$[u] + 1$
                        \State prev$[v] \gets u$
                        \State $Q$.push($v$)
                    \EndIf
                \EndFor
            \EndWhile
            \State \Return dist, prev
        \EndFunction
    \end{algorithmic}

    \textbf{Runtime Analysis}
    \begin{proof*}
        The first 3 operations are linear in $n$, and the last 4 are constant time. Then note that every vertex is added to the $Q$ once, and all of its edges are looped over.
        So the total runtime is:
        \[ T(n) \leq \Theta(n) + \sum_{v \in V} C \cdot (1 + \text{outdegree}(v)) \]
        Note that the sum of the outdegrees counts every edge once. Thus, $T(n) = \mathcal{O}(m + n)$.
    \end{proof*}

\end{algothm}

The interesting thing about BFS is that it is implemented the exact same as iterative DFS, but the stack is replaced with a queue.

To show correctness, we require a few intermediate results: 
\begin{theorem} [BFS Lemma 1]
    $\forall v \in V$, dist$[v] \leq \delta(s, v)$ (the true shortest distance).

    \begin{proof*}
        We proceed by induction on $k$, the number of push operations to $Q$ so far. $k = 1$ is trivial, since the dist of $s$ is 0 and everyone else's
        distance is infinity, satisfying the inequality. Consider the $k + 1$st push (with the induction holding for $\leq k$), during the edge $(u, v)$. We push $v$ when we visit $u$.
        By the inductive hypothesis, $\text{dist}[u] \geq \delta(s, u)$, so 
        \[ \text{dist}[v] = 1 + \text{dist}(u) \leq \delta(s, u) + 1 = \delta(s, u) + \delta(u, v) \geq \delta(s, v) \]
        completing the induction.
    \end{proof*}
\end{theorem}

\begin{theorem} [BFS Lemma 2]
    Look at any point in time $i$. Say $Q = [v_1, \dots, v_r]$. Then
    \begin{enumerate}
        \item $\forall i, \text{dist}[v_i] \leq \text{dist}[v_{i + 1}]$
        \item $\text{dist}[v_r] \leq \text{dist}[v_1] + 1$
    \end{enumerate}

    \begin{proof*}
        The proof is similar to the theorem above. Induct on the number of queue operations.
    \end{proof*}
\end{theorem}

Now we can show the correctness of BFS.

\begin{theorem}[The Correctness of Breadth-First Search]
    We will show that BFS finds shortest paths from $s$.

    \begin{proof*}
        For the sake of contradiction, suppose the dist arry is incorrect. Then, there is some $v \in V$ such that dist$[v] \neq \delta(s, v)$.
        By BFS Lemma 1, we must have dist$[v] > \delta(s, v)$. This may be the case for many such $v$. Let us pick the $v$ such that
        $\delta(s, v)$ is minimum (note that $v \neq s$). Then, let us take the shortest path from $s$ to $v$.
        \[ s \to v_1 \dots v_{r - 1} \dots v \]
        This means for all intermediate vertices until $v_{r - 1}$, the dist array was set correctly. Look at the point in time when $v_{r - 1}$ was popped
        off $Q$. But this means that $v$ was not put into the $Q$ (since the distance was resolved incorrectly) at this point. This means that $v$ was visited already by some
        other vertex $u$ and by BFS Lemma 2, this means that $\text{dist}[u] \leq \text{dist}[v_{r - 1}]$. But this would mean that $v$ got set to some value:
        \[ \text{dist}[v] = \text{dist}[u] + 1 \leq \text{dist}[v_{r - 1}] + 1 = \delta(s, v) \]
        However, we initially claimed dist$[v] > \delta(s,v)$, so this is a contradiction! So we cannot have any place where the dist array is incorrect.
    \end{proof*}
\end{theorem}


\subsubsection{Weighted Graphs}
Note that if all weights $w(e) \in \N$, then we can reduce finding the SSSP to the unweighted case by just subdividing edges into $w(e)$ fake vertices in the middle.
If $w(e) \leq L$, then the BFS runtime is $\mathcal{O}(n + mL)$. We can do better (and also use $w(e) \in \R$).

To do this, we use heaps (or priority queues). A min-heap is a data structure that maintains a set of (key, value) pairs $S$ subject to the following
three operations:
\begin{enumerate}
    \item delMin$()$, returns $(k, v)$ from $S$ with the smallest $k$ and removes it from $S$.
    \item decKey$(\*O, k')$, where $\*O$ is a pointer to the $v$ object, replaces the old key with a smaller key $k'$
    \item insert$(k, v)$, inserts $(k, v)$ into $S$
\end{enumerate}

We can now use these heaps to solve SSSPs for arbitrary non-negative weighted graphs.
\begin{algothm}[Dijkstra's Algorithm]
    \begin{algorithmic}
        \Function{dijkstra}{$G, s$}
            \State dist$[1\dots n] \gets \infty$
            \State prev$[1\dots n] \gets$ null
            \State $H \gets$ heap()
            \For{$v \in V$}
                \State $H$.insert($\infty, v$)
            \EndFor
            \State dist$[s] \gets 0$
            \State $H$.decKey($s, 0$)
            \While{$H$.size$>0$}
                \State $u \gets H$.delMin()
                \For{$(u, v) \in E$}
                    \If{dist$[u] + w((u, v)) < \text{dist}[v]$}
                        \State $H$.decKey($v, \text{dist}[u] + w(u, v))$
                        \State dist$[v] \gets$ dist$[u] + w((u, v))$
                        \State prev$[v] \gets u$
                    \EndIf
                \EndFor
            \EndWhile
            \State \Return dist, prev
        \EndFunction
    \end{algorithmic}

    \textbf{Runtime Analysis}
    We do $n$ insertions and thus must do $n$ deleteMins, and we are doing potentially $\deg(v)$ insertions for each, so overall, considering $t_I$ as the runtime of insert,
    $t_{dK}$ for decKey and $t_{dM}$ for delMin, we have the runtime is something close to:
    $T(n) = \mathcal{O}(n + m + nt_I + nt_{dM} + mt_{dK})$ which is:
    \[ T(n)  = \mathcal{O}((m + n) \log n) \]
    for a binary heap implementation.

    \textbf{Proof of Correctness}
    First realize the following: Any non $\infty$ dist$[u]$ value corresponds to some $s \to u$ path. This can be shown formally
    by induction on the main loop of the algorithm.

    The main claim is that the final dist and prev arrays are correct, so for all $u$, dist$[u] = \delta(s, u)$. Note that if we do this,
    then the prev array is correct because note that we change the prev arrays exactly when the dist arrays are set, so the path traced out
    is the correct one.

    To show the claim, let \[A = \{ \text{all vertices that have been popped off the heap so far} \}\]
    at the end of the algorithm, the heap is empty so all the vertices have been popped off. Thus, we will show the following invariant:
    that for any $u \in A$, dist$[u] = \delta(s, u)$. We proceed by induction.

    Base case: If $|A| = 1$, then only $s$ has been removed off the heap. Since we set its distance initially, we know dist$[s] = 0 = \delta(s, s)$.

    Inductive step: Say $|A| = k + 1$, so we just popped a vertex $v$ off the heap. So, we just need to make sure dist$[v]$ is correct; the rest are correct
    by inductive hypothesis ($|A| = k$ without $v$). Then, $v$ must've been the minimum key in the heap, so its dist must've been set last by some $u \in A$ (since it was popped off previously).
    Thus, there is an edge $(u, v)$. Furthermore, there is a path from $s$ to $u$ called $p$ by the inductive hypothesis. We claim that appending $(u, v)$ to $p$
    yields the shortest path. Suppose it wasn't and we could do better, with some other path $p'$. Then at some point, this path must leave $A$ (since $v$ was still outside $A$ at this point).
    Consider the last vertex before it leaves $A$ as $x$ and call the first vertex after it leaves as $y$. Let the subpath from $s$ to $x$ be $q$. Then, the length of $p'$ is as follows:
    \begin{align*}
        L(p') &\geq L(q) + w(x, y) \\
        &= \delta(s, x) + w(x, y) \\
        &= \text{dist}[x] + w(x, y) \\
    \end{align*} 
    But since $y$ is a neighbor of $y$, then we had to have updated it in the past, at least through the update at vertex $x$. Thus,
    the dist value is at most the last line, since that is what an update in dijkstra's looks like.
    \begin{align*}
        L(p') &\leq \text{dist}[y]
    \end{align*}
    However, since $v$ was the last thing popped off the heap and $y$ wasn't, its dist value is lower.
    \begin{align*}
        L(p') &\leq \text{dist}[v] \\
        &= L(p)
    \end{align*}
    Which means that $p'$ can't be any shorter than $p$
\end{algothm}
