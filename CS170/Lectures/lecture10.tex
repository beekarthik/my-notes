\subsection{Lecture 10}

\subsubsection{Union Find}

We now investigate how to build a union find data structure that
we described earlier (disjoint sets where find($\cdot$) tells you
the name of the set that some number is in, and union($\cdot$, $\cdot$)
unions together the associated sets). See the Kruskal's section for more information.

First, think of a naive approach. We could potentially store an array $A[1\dots n]$, where $A[x]$ tells you the name of $x$'s set.
find$(x)$ is easy, just return $A[x]$. This runs in constant time. To union$(x, y)$, we find $a = A[x]$, $b = A[y]$
and then change all of the indices having value $a$ to $b$ (by looping over them). This is linear time.

Secondly, let's try an array of linked lists. Represent $x$ by a linked list node.
$A[x]$ is a pointer to a linked list node that corresponds to $x$ (which is itself part of some circular doubly-linked list).
Thus, we end up with a collection of linked lists; make each of these one of the sets.

Then, to find$(x)$, we follow the pointer $A[x]$ all the way up its linked list to the head. This is worst-case
linear time. To union$(x, y)$, then put the head of one linked list to point to the tail of another linked list.
This also takes linear time to traverse, and then changing the pointers takes constant time.

Let us think of an optimization. How about if every node stored a pointer to its list's head node?
Well, during a union, we may need to update all the head pointers in some list; this would make union linear time
and find constant time. Another optimization is to make sure the smaller list always goes under the bigger list.
To do this, we can have the head pointers store their size.

Now the claim is that this optimized version, any sequence of operations
consisting of at most $m$ finds and at most $n$ unions takes total time $\mathcal{m + n \log n}$.
This is slower than the sorting step in Kruskal's and is actually enough to get optimal runtime.

\begin{proof*}
    First, doing a find takes constant time. But a single union takes:
    \[ \O(1) + \O(\text{head pointer changes}) \]
    Thus, the total runtime is:
    \[ \mathcal{O}(m + n + \text{total head pointer changes}) = \mathcal{O}(\sum_{x = 1}^n (\text{number of times I changed $x$'s' head pointer})) \]
    Note that whenever I change a head pointer, the size of the list that $x$ is a part of at least doubles.
    Thus, the maximum amount of times that list can double is $\log_2 n$, which will thus be at least amount of times I changed each node's head pointer.
    Thus, we end up with:
    \[ \mathcal{O}(m + n\log n) \]
    as we wanted to show.
\end{proof*}

However, this isn't the best we know. We know an even better algorithm, with
a disjoint forest. But first, we must digress to amortized runtime, to formalize
the idea of "good" average cost.

\begin{definition}[Amortized Runtime]
    Suppose a data structure supports operations $O_1, O_2, \dots, O_k$.
    Then, we say the amortized cost of each operation $t_j$ if for
    any sequence of operations with $N_i$ of operation $O_i$ over all $i$, the total time
    is
    \[ \leq \sum_{j = 1}^k t_j N_j \]
\end{definition}

Now, we look at the best way we know of implementing union find.

\begin{algothm}[Disjoint Forest Union Find]
    Let us represent sets as collections of rooted trees.
    First, we consider the path compression optimization. What this means is that whenever we traverse a find from some node to a root,
    then for every node we pass along the way, we make its pointer point directly to the root.
    
    Then, we do an analogous "length" optimization
    for union, where you make the shallower "list" (now trees) just point its root to the deeper tree's root (again, by storing the depth of the tree e.g.
    in an array). Then, to update the depth; it doesn't change if the depths of the trees are different, and if they are the same,
    then the depth of the root increases by 1.

    However, doing both of these optimizations creates a problem: we have to update the depth every time we do path compression.
    We just ignore updating the heights during a path compression (and maybe call this number rank instead of height).

    Here is the pseudo-code.

    $p[1\dots n]$, $r[1 \dots n]$ 
    \begin{algorithmic}
        \Function{MakeSet}{$x$}
            \State $p[x] \gets x$
            \State $r[x] \gets 0$
        \EndFunction
        \Function{Find}{$x$}
            \If{$x = p[x]$} 
            \State \Return $x$
            \EndIf
            \State $p[x] \gets$ \Call{Find}{$x$}
            \Return $p[x]$
        \EndFunction
        \Function{Union}{$x, y$}
            \State $x \gets$ \Call{Find}{$x$}
            \State $y \gets$ \Call{Find}{$y$}
            \If{$x = y$}
                \State \Return
            \EndIf
            \If{$r[x] > r[y]$}
                \State \Call{Union}{$x, y$}
            \EndIf
            \State $p[x] \gets y$
            \If{$r[x] = r[y]$}
                \State $r[y] \gets r[y] + 1$
            \EndIf
        \EndFunction
    \end{algorithmic}

    \textbf{Runtime Analysis} Let $\log^*(n)$ be the amount of times you have to take
    the $\log(n)$ before you get down to 1 (for all practical purposes, this is no more than 5).
    We claim any sequence of at most $m$ Finds and $n$ Unions/Makesets takes total time
    \[ \mathcal{O}((m + n) \log^*(n)) \]
    This gives us amortized $\log^*(n)$ runtime for all the operations.

    Before we do this, we claim some properties:
    \begin{enumerate}
        \item Any root node $x$ has $\geq 2^{r[x]}$ elements in its tree.
        \item Ranks strictly increase as you follow parent pointers.
        \item The number of nodes with rank exactly $k$ is $\leq \frac{n}{2^k}$.
    \end{enumerate}

    \begin{proof*}
        Imagine every number from 1 to $n$ either pays cash or charges its credit card (in terms of running time, every time it does an operation). Then,
        to find the total running time (cost) we add up this "cash" and add up all the credit card dues.

        Look at our $n$ items at any point in time. They all have some ranks; the ranks go from 0
        to $\log n$. We can place these items into buckets based on rank, in the following way:
        \[ [0, 1), [1, 2^1), [2, 2^{2}), [2, 2^{2^2}), \dots \]
        The largest rank that can exist is $\log_2(n)$ (by claims 1 and 2), so the number of groups is
        $\log^*(\log_2(n)) = \log^*(n) - 1$.

        Now let's do some "accounting." Find is generally what takes time; the total time is:
        \[ \mathcal{O}(m + n + \text{\# of parents pointers I follow}) \]
        Suppose we following a parent pointer from $u$ to $v$. Then, we have a few cases:
        \begin{enumerate}
            \item $v$ is the root of its tree. Let's pay cash here. In every find/union operation, we do this exactly once, so this is $\mathcal{O}(1)$ per op.
            \item $u, v$ are in different groups (as defined above) with neither as the root. We will again "pay cash".
            how many times do we change groups? Since rank increases as we follow parent pointers,
            so the maximum amount of times we can change groups while going up is $\mathcal{O}(\log^*(n))$ per op/
            \item $u, v$ are in the same groups (as defined above) with neither as the root. We charge these operations to $u$'s credit card; we will account for this at the end.
        \end{enumerate}
    \end{proof*}
\end{algothm}