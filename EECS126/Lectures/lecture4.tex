\subsection{Lecture 4}

\subsubsection{The Laws of Large Numbers, Revisited}
Here is a recap of the two different laws of large numbers.

First, we define two different types of convergence:
\begin{definition} [Almost Sure Convergence]
    A random variable $X_n$ almost surely converges to random variable $X$ if
    \[\Pr{X_n \to X} = 1 \]
    as $n \to \infty$.
\end{definition}

\begin{definition}[Convergence in Probability]
    A random variable $X_n$ converges in probability to random variable $X$ if
    \[ \Pr{\abs{X_n - X} > \epsilon} \to 0 \]
    for any real number $\epsilon > 0$ as $n \to \infty$.
\end{definition}

\begin{theorem} [Strong Law of Large Numbers]
    Let $X_1, X_2, \dots, X_n$ be independent and identically distributed (iid)
    random variables. Define:
    \[ Y_n = \frac{X_1 + \dots + X_n}{n} \]
    \[ Y = \E{X_1} \]
    $Y_n$ converges to $Y$ almost surely.
\end{theorem}

Note the contrast with the weak law of natural numbers. The weak law had only convergence in probability.
A key thing to note is that the strong law $\textbf{implies}$ the weak law.

\subsubsection{Independence}
Let us now refine the notion of Independence.

\begin{definition} [Pairwise Independence]
    Consider events $A_j$ with $j \in J$.

    The events are pairwise independent if for any $j, k \in J$,
    \[\Pr{A_j \cap A_k} = \Pr{A_j} \Pr{A_k}\]
\end{definition}

\begin{definition} [Mutual Independence]
    Consider events $A_j$ with $j \in J$.

    The events are mutually independent if 
    \[\Pr{\bigcap_{j \in K} A_j} = \prod_{j \in K} \Pr{A_j}, \forall K \subseteq J \]
\end{definition}

Note that pairwise independence does not imply mutual independence. Here is
an example of that edge case:

\begin{example}
    Take probability space $\Omega = \{1, 2, 3, 4\}$, all equally likely. Consider the events:
    $A = \{1, 2\}, B = \{1, 3\}, C=\{1,4\}$.

    Note that $\Pr{A \cap B} = \frac{1}{4} = \Pr{A} \Pr{B}$, 
    
    but
    $\Pr{A \cap B \cap C} = \frac{1}{4} \neq \frac{1}{8} = \Pr{A} \Pr{B} \Pr{C}$.
\end{example}

Now with independence, we can find that the converse of Borel-Cantelli is often true:

\begin{theorem} [Converse of Borel-Cantelli Theorem]
    Let $A_n$ be a collection of mutually indpendent events such that
    $\sum_{n = 1}^{\infty} \Pr{A_n} = \infty$. Then, $\Pr{A_n \text{ infinitely often}} = 1$.
\end{theorem}

Let us use another example to understand this:

\begin{example}
    Let $A_n$ be the same event as the other example (the nth flip is heads) and assign:
    \[ \Pr{A_n} = \frac{1}{n} \]
    where all the $A_n$ are mutually independent.

    Since $\sum_{n = 1}^{\infty} \frac{1}{n} = \infty$ and thus by the converse of Borel-Cantelli:
    $\Pr{A_n \text{ io}} = 1$.
\end{example}

\begin{example}[Glued Coins]
    Suppose you have $n$ coins that are all glued together, i.e. the only two outcomes
    are $HHH\dots$ or $TTT\dots$. Then let $A_n = {\text{the $n$th coin is heads}}$. Note that
    \begin{align*}
        \Pr{A_n} &= \frac{1}{2} \\
    \end{align*}
    which means $\sum \Pr{A_n} = \infty$. Thus the probability of getting a sequence which has heads
    infinitely often is 1.
\end{example}

\begin{theorem}[Kolmogorov's 0-1 theorem]
    If you have a set of events $\{A_n\}_{n = 1}^{\infty}$ that all independent,
    then
    \[ \Pr{A_n \text{ infinitely often}} = 0 \text{ or } 1 \]
\end{theorem}

\subsubsection{Conditional Probability}
Now we refine conditional probability for many events.

\begin{definition} [Conditional Probability]
    Let $A$ and $B$ be two events, and assume $\Pr{B} > 0$. Then the conditional probability of $A$ given $B$ is:
    \[ \Pr{A \mid B} = \frac{\Pr{A \cap B}}{\Pr{B}} \]
\end{definition}

\begin{theorem} [Chain Rule]
    For two events we had $\Pr{A \cap B} = \Pr{A \mid B} \Pr{B}$. For $n$ events
    $A_i$, we have:
    \[ \Pr{A_1 \cap A_2 \cap \dots \cap A_n} = \Pr{A_1} \Pr{A_2 \mid A_1} \Pr{A_3 \mid A_1 \cap A_2} \dots \Pr{A_n \mid A_1 \cap A_2 \cap \dots \cap A_{n -1}} \]
    if $\Pr{A_1 \cap A_2 \cap \dots A_{n - 1}} > 0$.
\end{theorem}

The generalized result above can be shown by induction, taking the case of two events as the base case and then inducting on $n$.
Now we will bring in some of the most powerful tools.

\begin{theorem}[Law of Total Probability]
    Let $A, B_1, \dots B_n$ be events where $B_i$'s are disjoint and $\bigcup_{i=1}^n B_i = \Omega$.
    Then,
    \[ \Pr{A} = \sum_{i = 1}^n \Pr{A \cap B_i} \]
\end{theorem}

\begin{theorem} [Bayes' Rule]
    Let $A, B_1, \dots B_n$ be events where $B_i$'s are disjoint and $\bigcup_{i=1}^n B_i = \Omega$.
    \[ \Pr{B_i \mid A} = \frac{\Pr{A \mid B_i} \Pr{B_i}}{\sum_{j = 1}^n \Pr{A \mid B_j}\Pr{B_j}} \]

    \begin{proof*}
        Note that we can use the initial definition to expand the left side:
        \begin{align*}
            \Pr{B_i \mid A} &= \frac{\Pr{A \cap B_i}}{\Pr{A}} \\
            &= \frac{\Pr{A \mid B_i} \Pr{B_i}}{\sum_{j = 1}^n \Pr{A \cap B_j}} \\
            &= \frac{\Pr{A \mid B_i} \Pr{B_i}}{\sum_{j = 1}^n \Pr{A \mid B_j} \Pr{B_j}} \\
        \end{align*}
        where the summation in the denominator comes from the law of total probability.
    \end{proof*}
\end{theorem}

Often the $B_j$'s are termed the prior probabilities, and $A$ is considered the posterior probability.

For an event $B \subseteq R$, $\Pr{X \in B} = \Pr{(X^{-1}(B))}$ where
\[ X^{-1}(B) = \{ \omega \in \Omega \mid X(\omega) \in B \} \].

We can define the following for a random variable to the reals.
\begin{definition} [Cumulative Distribution Function (CDF)]
    The Cumulative Distribution Function $F_X(x)$ of random variable $X$ is defined by:
    \[ F_X(x) = \Pr{X \in (- \infty, x]} = \Pr{X \leq x} \]
\end{definition}

Here are some properties of the CDF:
\begin{itemize}
    \item $F_X$ is non-decreasing.
    \item $F_X$ is right-continuous.
    \item $F_X \to 0$ as $x \to -\infty$ and $F_X \to 1$ as $x \to \infty$.
\end{itemize}

\begin{example} [CDF of an Indicator]
    Consider the following random variable:
    \[ I = \begin{cases}
        0 \text{ with probability $1-p$} \\
        1 \text{ with probability $p$}
    \end{cases} \]
    Then the $F_I(i)$ is a step function:
    TODO Add figure
\end{example}